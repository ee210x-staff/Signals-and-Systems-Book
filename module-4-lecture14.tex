\section{Module 4: Lecture 14\\Signal Flow Graphs}

\tikzset{
	roundnode/.style={coordinate},
	squarednode/.style={rectangle, draw=red!60, fill=red!5, very thick, minimum size=2mm}
}

\subsection{Introduction}
We've been doing a lot of formal constructions and formal derivations in the previous few sessions. To relate these formalisms to systems now, and getting a feel of how we can actually understand system construction, at least, for rational systems if not irrational, we will introduce the method of drawing a block diagram of a system and a way to connect it to the system description. These are called \textit{signal flow graphs}.\\
In a signal flow graph, we would like to realize the system using basic building components. This will be best understood if we study it using an example.
\subsection{Continuous Independent Variable Case}
Let us assume that we have a rational system with system function given by
\[
H(s) = \frac{s+1}{s+2} \ \text{with ROC given by } Re(s)>-2
\]
We can write this as
\[ \frac{Y(s)}{X(s)} = \frac{s+1}{s+2} \]
\[ (s+2)Y(s) = (s+1)X(s) \]
\begin{equation}
Y(s) = -\frac{1}{2}s Y(s) + \frac{1}{2} s X(s) + \frac{1}{2}X(s)
\end{equation}
Now, we can invert this Laplace transforms to get,
\[ y(t) = -\frac{1}{2}\frac{\dm y(t)}{\dm t} + \frac{1}{2}\frac{\dm x(t)}{\dm t} + \frac{1}{2}x(t) \]
In this form, we can see what basic components are needed to generate the output.
\begin{itemize}
\item Adder: For adding two signals (indicated by the $+$ sign in the equation).
\item Differentiator: For differentiating with respect to the independent variable (indicated by the $\frac{\dm}{\dm t}$ operator in the equation).
\item Constant Multipliers: (Half, in this case).
\end{itemize}
Now, having identified the basic components needed, we can draw the signal flow graph. The signal flow graph consists of two elements:
\begin{enumerate}
\item Nodes: These contain the Laplace transformed (or Z-transformed, in case of discrete independent variable case) signals and their combinations.
\item Directed Edges: These are the transportation systems, with the multipliers embedded in them.
\end{enumerate}
The meaning of these will be clear with this example. Consider the right hand side of the equation (1). The $X(s)$ dependent terms are  $\frac{1}{2} s X(s)$ and $\frac{1}{2}X(s)$. Now, these will be generated by the following flow graph.\\
\begin{center}
	\begin{tikzpicture}
		\matrix (m1) [row sep=15mm, column sep=15mm]
		{
			\node[dspnodeopen,dsp/label=left]  (m00) {$X(s)$}; &
			\node[dspnodeopen]         (m01) {};       &
			\node[dspnodeopen]         (m02) {};       &
			\node[dspnodeopen,dsp/label=right] (m03) {$\frac{1}{2}(s+1)X(s)$};        \\
		%--------------------------------------------------------------------
			\node[coordinate]                  (m10) {};        &
			\node[dspnodeopen]                 (m11) {};        \\
		};
% Draw connections
	
		\draw[dspflow] (m00) -- node[midway,above] {$1/2$} (m01);
	
		\draw[dspflow] (m01) -- (m02);
	
		\draw[dspflow] (m02) -- (m03);

		\draw[dspflow] (m01) -- node[midway,left] {$s$} (m11);
		
		\draw[dspflow] (m11) -- (m02);
	\end{tikzpicture}
\end{center}
The circles are the nodes and the rays are the directed edges. The directed edges which don't have a multiplier associated are the unity multipliers, which essentially just carry the signal from one point to another. Whereas those who have a multiplier multiply the signal by the corresponding value and then direct it. This multiplier can either be constant or $s$ dependent, as can be seen in the figure.\\
Now, suppose that $Y(s)$ is generated somewhere. We will denote this by\\
\begin{center}
	\begin{tikzpicture}
	\matrix (m1) [row sep=15mm, column sep=15mm]
	{
		\node[dspnodeopen]  (m00) {};  &
		\node[dspnodeopen, dsp/label=right]  (m01) {$Y(s)$}; \\
	};
	\draw[dspflow] (m00) -- (m01);
\end{tikzpicture}
\end{center}
Note that both its ends are $Y(s)$. Now we need to add $-\frac{1}{2}sY(s)$ to $-\frac{1}{2}(s+1)X(s)$ to get $Y(s)$. To do that, we will have generate $\frac{1}{2}sY(s)$. It is shown as
\begin{center}
	\begin{tikzpicture}
		\matrix (m1) [row sep=15mm, column sep=15mm]
		{
			\node[dspnodeopen, dsp/label=left]  (m00) {$-\frac{1}{2}sY(s)$};  &
			\node[dspnodeopen]  (m01) {};  &
			\node[dspnodeopen]  (m02) {};  &
			\node[dspnodeopen, dsp/label=right](m03){$Y(s)$}; \\
			
			\node[coordinate]   (m10) {};  &
			\node[dspnodeopen]  (m11) {};  &
			\node[dspnodeopen]  (m12) {};  &
			\node[coordinate]   (m13) {};  \\
		};
		\draw[dspflow] (m00) -- (m01);
		\draw[dspflow] (m01) -- (m02);
		\draw[dspflow] (m02) -- (m03);
		\draw[dspflow] (m11) -- (m01);
		\draw[dspflow] (m12) -- node[midway,below] {$-1/2$}(m11);
		\draw[dspflow] (m02) -- node[midway,right] {$s$}(m12);
	\end{tikzpicture}
\end{center}
Note that this graph has to be read from right to left. $Y(s)$ is first multiplied by half, then by $s$ to generate $\frac{1}{2}sY(s)$ at the left. Now, to add this to the previous graph to complete the signal flow graph, we just join the two:
\begin{center}
	\begin{tikzpicture}
		\matrix (m1) [row sep=15mm, column sep=15mm]
		{
			\node[dspnodeopen,dsp/label=left](m00){$X(s)$};&
			\node[dspnodeopen]               (m01) {};     &
			\node[dspnodeopen] 				 (m02) {};     &
			\node[dspnodeopen] 				 (m03) {};     &
			\node[dspnodeopen] 				 (m04) {};     &
			\node[dspnodeopen,dsp/label=right](m05){$Y(s)$}; \\	%--------------------------------------------------------------------
			\node[coordinate]                (m10) {};     &
			\node[dspnodeopen]               (m11) {};     &
			\node[coordinate]                (m12) {};     &
			\node[dspnodeopen]               (m13) {};     &
			\node[dspnodeopen]               (m14) {};     &
			\node[coordinate]                (m15) {};      \\
		};
% Draw connections
	
		\draw[dspflow] (m00) -- node[midway,above] {$1/2$} (m01);
	
		\draw[dspflow] (m01) -- (m02);
		
		\draw[dspflow] (m02) -- (m03);
		
		\draw[dspflow] (m03) -- (m04);
		
		\draw[dspflow] (m04) -- (m05);

		\draw[dspflow] (m01) -- node[midway,left] {$s$} (m11);
		
		\draw[dspflow] (m11) -- (m02);
		
		\draw[dspflow] (m13) -- (m03);
		
		\draw[dspflow] (m14) -- node[midway,below] {$-1/2$} (m13);
		
		\draw[dspflow] (m04) -- node[midway,right] {$s$} (m14);
\end{tikzpicture}
\end{center}
We now have the complete signal flow graph for the system.\\
As we can see in the graph, there are three types of nodes.
\begin{enumerate}
\item Nodes from which lines go outward, but no line comes inward. Such nodes are called \textit{source nodes}. Here, the node containing $X(s)$ is the source node.
\item Nodes in which lines come inward, but no lines go outward. Such nodes are called \textit{sink nodes}. Here, the node containing $Y(s)$ is the sink node.
\item Nodes in which some lines come inward and some go outward. Such nodes are called \textit{intermediate nodes}. Here, all the nodes apart from the source and the sink nodes are intermediate nodes.
\end{enumerate}
Notice that we have a loop here. That means you have a set of directed edges which start at the given node and reach back the same node. This happens because the output $Y(s)$ is itself dependent on $Y(s)$, which indicates that there is a recursion in the system. Loops are bound to be present in recursive systems.\\
Note that here we have used differentiators (multiplication by $s$) to realize the system. In general, it is not a good idea to use differentiation in continuous independent variable case. This is because the system function of a differentiator is $H(s)=s$. Now, the frequency response of the differentiator is  given by $H(\Omega)=j\Omega$ (putting $Re(s)=0$). We can see that the modulus of this increases linearly with $\Omega$, that is, linearly with the frequency. Hence, a differentiator over-emphasizes higher frequencies, and the noise in the system generally exists in the higher frequencies. Hence, if a differentiator is used, the noise will get enhanced, which is not desired.\\
Instead, we can use an integrator to realize the same system. The system function for the integrator is $H(s)=\frac{1}{s}$ and the frequency response is $H(\Omega)=\frac{1}{j\Omega}$. Now this system attenuates the higher frequencies. Hence, an integrator is preferred over a differentiator. Although the integrator over-emphasizes constant inputs, any other kind of inputs don't have a problem.\\
To realize this system using integrators, we write
\[
H(s) = \frac{Y(s)}{X(s)} = \frac{s+1}{s+2}=\frac{s^{-1}}{s^{-1}}\frac{s+1}{s+2} = \frac{1+s^{-1}}{1+2s^{-1}}
\]
Hence, we can write
\[
Y(s) = -2s^{-1}Y(s) + (1+s^{-1})X(s)
\]
and we can correspondingly draw the signal flow graph as follows:
\begin{center}
	\begin{tikzpicture}
		\matrix (m1) [row sep=15mm, column sep=15mm]
		{
			\node[dspnodeopen,dsp/label=left](m00){$X(s)$};&
			\node[dspnodeopen]               (m01) {};     &
			\node[dspnodeopen] 				 (m02) {};     &
			\node[dspnodeopen] 				 (m03) {};     &
			\node[dspnodeopen] 				 (m04) {};     &
			\node[dspnodeopen,dsp/label=right](m05){$Y(s)$}; \\	%--------------------------------------------------------------------
			\node[coordinate]                (m10) {};     &
			\node[dspnodeopen]               (m11) {};     &
			\node[coordinate]                (m12) {};     &
			\node[dspnodeopen]               (m13) {};     &
			\node[dspnodeopen]               (m14) {};     &
			\node[coordinate]                (m15) {};      \\
		};
% Draw connections
	
		\draw[dspflow] (m00) -- node[midway,above] {} (m01);
	
		\draw[dspflow] (m01) -- (m02);
		
		\draw[dspflow] (m02) -- (m03);
		
		\draw[dspflow] (m03) -- (m04);
		
		\draw[dspflow] (m04) -- (m05);

		\draw[dspflow] (m01) -- node[midway,left] {$s^{-1}$} (m11);
		
		\draw[dspflow] (m11) -- (m02);
		
		\draw[dspflow] (m13) -- (m03);
		
		\draw[dspflow] (m14) -- node[midway,below] {$-2$} (m13);
		
		\draw[dspflow] (m04) -- node[midway,right] {$s^{-1}$} (m14);
\end{tikzpicture}
\end{center}
As you can see, there are more than one ways of drawing the signal flow graph for a given system. The defining equation is the same, the system function is the same, but we have different structures. Now which structure is better, which structure is worse, is a deep subject by itself. Very often we would prefer integrators instead of differentiators in continuous variable systems. However that is also not universally true. There could be some situations where you want a differentiator, for one reason or the other. Or you may want a combination of differentiators and integrators, and many structures are possible, as it is seen in this case.\\
Let us see how this construction translates in the discrete independent variable case.
\subsection{Discrete Independent Variable Case}
We will again understand this using an example. Let the system function of a discrete system be given by
\[
H(z) = 	\frac{1+z^{-1}}{1-\frac{1}{2}z^{-1}} \text{ with ROC as } |z|>\frac{1}{2}
\]
Hence, we have
\[
(1-\frac{1}{2}z^{-1})Y(z) = (1+z^{-1})X(z)
\]
We can invert the Z-transforms, giving
\[
y[n]-\frac{1}{2}y[n-1] = x[n]+x[n-1]
\]
We can rearrange this to get
\[
y[n]=\frac{1}{2}y[n-1] + x[n]+x[n-1]
\]
This is an example of a Linear Constant Coefficient \emph{Difference} Equation. Difference essentially means shifted input and output. They can be once or more delayed functions, of the form $x[n-M]$ or $y[n-M]$ for some natural number $M$. Higher delays can arise due to terms like $z^{-2}$ and so on. We are considering only delayed versions and not advanced versions, to ensure causality. Now, to realize a delay of the form $x[n-M]$ for some $M>1$, we can use unit delays $M$ times. Hence, a unit delay is a basic building component.\\
Now, coming back to our example, we can write $Y(z)$ as
\[
Y(z) = \frac{1}{2}z^{-1}Y(z)+(1+z^{-1})X(z)
\]
Analogous to the continuous independent variable case, we can draw a signal flow graph for this system. In this case, the nodes are the Z-transforms of the sequences. To generate the $X(z)$ dependent term, we can draw the following graph.
\begin{center}
	\begin{tikzpicture}
		\matrix (m1) [row sep=15mm, column sep=15mm]
		{
			\node[dspnodeopen,dsp/label=left]  (m00) {$X(z)$}; &
			\node[dspnodeopen]         (m01) {};       &
			\node[dspnodeopen]         (m02) {};       &
			\node[dspnodeopen,dsp/label=right] (m03) {$(1+z^{-1})X(z)$};        \\
		%--------------------------------------------------------------------
			\node[coordinate]                  (m10) {};        &
			\node[dspnodeopen]                 (m11) {};        \\
		};
% Draw connections
	
		\draw[dspflow] (m00) -- node[midway,above] {} (m01);
	
		\draw[dspflow] (m01) -- (m02);
	
		\draw[dspflow] (m02) -- (m03);

		\draw[dspflow] (m01) -- node[midway,left] {$z^{-1}$} (m11);
		
		\draw[dspflow] (m11) -- (m02);
	\end{tikzpicture}
\end{center}
Now, as the nodes are the Z-transforms, multiplying by $z^{-1}$ amounts to delaying the sequence by one unit.\\
We can similarly draw the $Y(z)$ dependent term, and finally get the signal flow graph as follows:
\begin{center}
	\begin{tikzpicture}
		\matrix (m1) [row sep=15mm, column sep=15mm]
		{
			\node[dspnodeopen,dsp/label=left](m00){$X(z)$};&
			\node[dspnodeopen]               (m01) {};     &
			\node[dspnodeopen] 				 (m02) {};     &
			\node[dspnodeopen] 				 (m03) {};     &
			\node[dspnodeopen] 				 (m04) {};     &
			\node[dspnodeopen,dsp/label=right](m05){$Y(z)$}; \\	%--------------------------------------------------------------------
			\node[coordinate]                (m10) {};     &
			\node[dspnodeopen]               (m11) {};     &
			\node[coordinate]                (m12) {};     &
			\node[dspnodeopen]               (m13) {};     &
			\node[dspnodeopen]               (m14) {};     &
			\node[coordinate]                (m15) {};      \\
		};
% Draw connections
	
		\draw[dspflow] (m00) -- node[midway,above] {} (m01);
	
		\draw[dspflow] (m01) -- (m02);
		
		\draw[dspflow] (m02) -- (m03);
		
		\draw[dspflow] (m03) -- (m04);
		
		\draw[dspflow] (m04) -- (m05);

		\draw[dspflow] (m01) -- node[midway,left] {$z^{-1}$} (m11);
		
		\draw[dspflow] (m11) -- (m02);
		
		\draw[dspflow] (m13) -- (m03);
		
		\draw[dspflow] (m14) -- node[midway,below] {$1/2$} (m13);
		
		\draw[dspflow] (m04) -- node[midway,right] {$z^{-1}$} (m14);
\end{tikzpicture}
\end{center}
Now, we will look at what is the general form of output a rational system gives to an input with a rational Laplace transform or Z-transform.
\subsection{General Form of Rational System Outputs}
If the input to a rational system has a rational Laplace transform or Z-transform, then the output will also have a rational Laplace transform or Z-transform. According to the general procedure, we will segregate the poles of the output by decomposing it in to partial fractions, have a look at the region of convergence and then identify the term corresponding to each poles viz. the ROC.\\
Now, the output poles will be of three types, in general:
\begin{enumerate}
\item Input poles which are \emph{not} system poles.
\item System poles which are \emph{not} input poles.
\item Coincident system and input poles.
\end{enumerate}
The third case can arise if the input and the system have some poles in common. The terms arising due to each of these poles are given a specific name. The terms due to the input poles are called the \emph{forced response} terms, the ones due to the system poles are called the \emph{natural response} terms and the ones due to the coincident poles are called the \emph{resonant response} terms. Thus, there are three kinds of terms in general, although one of them may be absent in specific cases. So, the forced response will have terms drawn from the input, whereas the natural response will have terms drawn from the impulse response, and of course the coefficients will vary according to the partial fraction expansion.\\
A physical example for the resonant response is the well known fact that bridges can collapse if many people walk on it in rhythm. The bridge can be approximated to have a rational system function with simple poles. For simple poles, the corresponding terms are just sinusoidal, and won't affect the bridge. But if many people walk on it in rhythm, that may generate a strong enough input, possibly having poles at the exact same place as that of the system function of the bridge. Due to this, the system pole can become double or triple, in which case the corresponding term will be a sinusoid multiplied by time, or the square of time, etc. Hence, we will have ever increasing oscillations, which can damage the bridge, even causing it to collapse completely.
\subsection{Conclusion}
We have laid the foundation for being able to deal with at least a reasonably good class of systems and inputs. We've been able to capture the essence of the transform method, generalised also to unstable systems as you can say, in the Laplace domain and in the Z domain respectively for continuous independent variable and discrete independent variable systems.\\
Looking at the whole module, we have generalised the transformed domain of module two, and also to carried forth certain ideas that we had learnt in module one and module three.\\
This transformed domain is more general. By replacing $s$ by $j\Omega$, we can go back to the Fourier Transform from the Laplace domain in continuous independent variable case, and similarly by replacing $z$ by $e^{j\Omega}$ in the discrete independent variable case, to get the Discrete Time Fourier Transform. These techniques that you've learnt in all these four modules and particularly in this fourth module are used in many branches of engineering, not just electrical engineering or mechanical engineering or systems engineering or control.

