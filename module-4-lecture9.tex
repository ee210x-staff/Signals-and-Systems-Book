\section{Module 4: Lecture 9\\Region of Convergence(ROC), Convolution Property and System Function}


\subsection{Differentiation property of \textit{z-transform}}
Suppose we have a sequence x[n] and
\[
x[n]\ \xrightarrow{\ \mathcal{Z} \ }\ X(z)\ ;\ ROC:\ R_X
\]
\[
X(z)\ =\ \sum\limits_{n=-\infty}^{\infty}x[n]\cdot z^{-n}
\]
Let us consider $\frac{d}{dz}X(z)$ with the same ROC, except for some possible boundaries.
\[
X(z)\ =\ \sum\limits_{n=-\infty}^{\infty}x[n]\cdot z^{-n}
\]
Here we will differentiate term by term, and if the derivative converges
\[
\frac{d}{dz}X(z)\ =\ \sum\limits_{n=-\infty}^{\infty}x[n]\cdot(-n)\cdot z^{-n-1}
\]
Now, multiply both sides by ($-z$)
\[
(-z)\cdot\frac{d}{dz}X(z)\ =\ \sum\limits_{n=-\infty}^{\infty}x[n]\cdot(n)\cdot z^{-n}
\]
The R.H.S. is essentially the \textit{z-transform} of $'n\cdot x[n]'$. Therefore,
\[
x[n]\ \xrightarrow{\ \mathcal{Z} \ }\ X(z);\ ROC: R_X
\]
\[
n\cdot x[n]\ \xrightarrow{\ \mathcal{Z}\ }\ (-z)\frac{d}{dz}X(z);
\]
\[
ROC: essentially\ the\ same\ ROC\ as\ X(z)\ with\ some\ boundaries
\]\\
Let us take an example
\[
(\frac{1}{2})^n\cdot u[n]\ \xrightarrow{\ \mathcal{Z} \ }\ \frac{1}{1 - \frac{1}{2}z^{-1}};\ |z|\ >\ \frac{1}{2}
\]
\[
n\cdot (\frac{1}{2})^n\cdot u[n]\ \xrightarrow{\ \mathcal{Z} \ }\ (-z)\cdot\frac{d}{dz}(\frac{1}{1 - \frac{1}{2}z^{-1}});\ |z|\ >\ \frac{1}{2} 
\]
\[
(-z)\cdot\frac{d}{dz}(\frac{1}{1 - \frac{1}{2}z^{-1}})\ =\ \frac{(-z)\cdot(-1)\cdot(-\frac{1}{2})(z^{-2})}{(1 - \frac{1}{2}z{-1})^2}\ =\ \frac{\frac{1}{2}z^{-1}}{(1 - \frac{1}{2}z{-1})^2}
\]
\[
\Rightarrow\ n\cdot (\frac{1}{2})^n\cdot u[n]\ \xrightarrow{\ \mathcal{Z} \ }\ \frac{\frac{1}{2}z^{-1}}{(1 - \frac{1}{2}z{-1})^2};\ ROC:\ |z|\ >\ \frac{1}{2}
\]
Here, we have a square of the denominator with some numerator.\\
This gives the methodology of dealing with repeated factors in the denominator in the \textit{z-transform}. Its principles are the same as that in the \textit{Laplace transform}.\\\\
Let us see the effect of convolution in \textit{z-domain}
\[
x[n]\ \xrightarrow{\ \mathcal{Z} \ }\ X(z);\ ROC:\ R_X
\]
\[
h[n]\ \xrightarrow{\ \mathcal{Z} \ }\ H(z);\ ROC:\ R_H
\]
\[
(x*h)[n]\ \xrightarrow{\ \mathcal{Z} \ }\ X(z)\cdot H(z);\ ROC:\ atleast\ R_X \cap  R_H 
\]
\subsection{How to handle rational \textit{z-transforms}?}
Now, with the convolution property and the differentiation property of the \textit{z-transforms}, we can solve all rational \textit{z-transforms}.\\\\
Let us take an example.\\ \\
Let $|\alpha|$ < $|\beta|$ and\\
\[
Y(z)\ =\ \frac{1}{(1 - \alpha z^{-1})(1 - \beta z^{-1}))};\ ROC:\ |z| > max(|\alpha|,|\beta|)\ or\ |z| > |\beta|
\]
Now this can be solved by two methods.\\\\
Method 1:\\\\
First, decompose $Y(z)$ into partial fractions.
\[
\frac{1}{(1 - \alpha z^{-1})(1 - \beta z^{-1})}\ =\ \frac{A}{(1 - \alpha z^{-1})}\ +\ \frac{B}{(1 - \beta z^{-1})}
\]
To find $'A'$, multiply both side by $(1 - \alpha z^{-1})(1 - \beta z^{-1})$ and put z = $\alpha$ \\
\[
\Rightarrow \ A\ =\ \frac{1}{1 - \beta\alpha^{-1}}
\]
Similarly, to find $'B'$, multiply both side by $(1 - \alpha z^{-1})(1 - \beta z^{-1})$ and put z = $\beta$ \\
\[
\Rightarrow \ B\ =\ \frac{1}{1 - \alpha\beta^{-1}}
\]
Therefore,
\[
Y(z)\ =\ \frac{\frac{1}{1 - \beta\alpha^{-1}}}{1 - \alpha z^{-1}}\ +\ \frac{\frac{1}{1 - \alpha\beta^{-1}}}{1 - \beta z^{-1}}
\]
\[
Y(z)\ =\ \frac{\frac{\alpha}{\alpha - \beta}}{1 - \alpha z^{-1}}\ -\ \frac{\frac{\beta}{\alpha - \beta}}{1 - \beta z^{-1}}
\]
\[
Y(z)\ =\ \frac{1}{(\alpha - \beta)}\cdot (\frac{\alpha}{1 - \alpha z^{-1}}\ -\ \frac{\beta}{1 - \beta z^{-1}});\ ROC:\ |z| > |\beta|\ which\ also\ implies\ |z| > |\alpha|
\]
Also $'A'$ and $'B'$ can be found by multiplying both sides by $(1 - \alpha z^{-1})(1 - \beta z^{-1})$, which gives
\[
A\cdot(1 - \beta z^{-1})\ +\ B\cdot(1 - \alpha z^{-1})\ =\ 1
\]
\[
(A + B)\ -\ (A\beta + B\alpha)\ =\ 1
\]
\[
\Rightarrow A\ +\ B\ =\ 1\ and
\]
\[
A\beta\ +\ B\alpha \ =\ 0
\]
Solving both the equations we get,\\
\[
A\ =\ \frac{1}{1 - \beta\alpha^{-1}},\ B\ =\ \frac{1}{1 - \alpha\beta^{-1}}
\]
or
\[
A\ =\ \frac{\alpha}{\alpha - \beta},\ B\ =\ \frac{-\beta}{\alpha - \beta}
\]
Now,
\[
\frac{\alpha}{\alpha - \beta}\ \xrightarrow{\ \mathcal{Z}^{-1}\ }\ \alpha \cdot \alpha^n \cdot u[n]\ =\ \alpha^{n+1}\cdot u[n]
\]
and
\[
\frac{\beta}{\alpha - \beta}\ \xrightarrow{\ \mathcal{Z}^{-1}\ }\ \beta \cdot \beta^n \cdot u[n]\ =\ \beta^{n+1}\cdot u[n]
\]
Therefore,
\[
Y(z)\ \xrightarrow{\ \mathcal{Z}^{-1} \ }\ y[n]
\]
and
\[
y[n]\ =\ \frac{(\alpha^{n+1} - \beta^{n+1})}{(\alpha - \beta)}u[n]
\]
or,
\[
y[n]\ =\ \frac{(\beta^{n+1} - \alpha^{n+1})}{(\beta - \alpha)}u[n]
\]
\\ Method 2: \\\\
By convolution property of \textit{z-transform}
\[
Y(z)\ =\ \frac{1}{(1 - \alpha z^{-1})(1 - \beta z^{-1})}\ =\ \frac{1}{(1 - \alpha z^{-1})}\cdot \frac{1}{(1 - \beta z^{-1})}
\]
For \[\frac{1}{(1 - \alpha z^{-1})(1 - \beta z^{-1})};\ ROC:\ |z|\ >\ |\beta|\]
For \[\frac{1}{(1 - \alpha z^{-1})};\ ROC:\ |z| > |\alpha|\]
For \[\frac{1}{(1 - \beta z^{-1})};\ ROC:\ |z| > |\beta|\]
So, \[\frac{1}{(1 - \alpha z^{-1})};\ ROC:\ |z| > |\alpha|\ \xrightarrow{\ \mathcal{Z}^{-1}\ }\ \alpha^n \cdot u[n] 
\]
Similarly, \[\frac{1}{(1 - \beta z^{-1})};\ ROC:\ |z| > |\beta|\ \xrightarrow{\ \mathcal{Z}^{-1}\ }\ \beta^n \cdot u[n] 
\]
Therefore, \[\frac{1}{(1 - \alpha z^{-1})(1 - \beta z^{-1})};\ ROC:\ |z| > |\beta|\ \xrightarrow{\ \mathcal{Z}^{-1}\ }\ \sum\limits_{k=-\infty}^{\infty}\alpha^k\cdot u[k]\cdot\beta^{n-k}u[n-k] \]
$u[k] = 1$, only for $n \geq  0$\\
$u[n-k] = 1$, only for $n-k \geq  0$\\
$u[k]\cdot u[n-k] = 1$, only for $0 \leq k\leq 0$\\
\[\Rightarrow y[n]\ =\ (\sum\limits_{k=0}^{n}\alpha^k\beta^{n-k})\]
Here $u[n]$ comes from the fact that only for $n\geq 0$, can $\ 0 \leq k\leq n$  be a non-trivial interval.
\[
y[n]\ =\ \beta^n\cdot u[n]\sum\limits_{k=0}^{n}\alpha^k\beta^{-k}
\]
\[
y[n]\ =\ \beta^n\cdot u[n]\sum\limits_{k=0}^{n}(\alpha\beta^{-1})^k
\]
\[
y[n]\ =\ \beta^n\cdot u[n](\frac{1\ -\ (\alpha\beta^{-1})^{n+1}}{1\ -\ \alpha\beta^{-1}})
\]
\[
y[n]\ =\ \beta^{n+1}\cdot u[n](\frac{1\ -\ (\alpha\beta^{-1})^{n+1}}{\beta\ -\ \alpha})
\]
\[
y[n]\ =\ (\frac{\beta^{n+1}\ -\ \alpha^{n+1}}{\beta\ -\ \alpha})
\]
The results from both the methods are same.
\subsection{\textit{z-transforms} with repeated factors}
Let us take an example.
\[
H(z)\ =\ \frac{1}{(1 - \frac{1}{2}z^{-1})^2\cdot(1-\frac{1}{3}z^{-1})}\ :\ ROC: |z| > \frac{1}{2}
\]
and we need to find the inverse \textit{z-transform}.\\
So the first step is to decompose \textit{H(z)} into partial fractions.
\[
H(z)\ =\ \frac{A_{11}}{1 - \frac{1}{2}z^{-1}} + \frac{A_{21}}{(1 - \frac{1}{2}z^{-1})^2} + \frac{A_{2}}{(1 - \frac{1}{3}z^{-1})}
\]
\[
=\ \frac{1}{(1 - \frac{1}{2}z^{-1})^2\cdot(1-\frac{1}{3}z^{-1})}
\]
Here we see that the repeated factor occur in all its powers.\\\\
Now to find $A_{21}$ multiply both sides by $(1 - \frac{1}{2}z^{-1})^2$ and put \textit{z} = $\frac{1}{2}$.\\
Solving this we get, $A_{21}$ = 3.\\\\
$A_{2}$ can be found by multiplying both sides by $(1 - \frac{1}{2}z^{-1})$ and put \textit{z} = $\frac{1}{3}$\\
Solving this we get $A_{2}$ = 4.\\\\\
To find $A_{11}$, solve this\\
\[
A_{11}\ =\ (1 - \frac{1}{2}z^{-1})\cdot(\frac{1}{(1 - \frac{1}{2}z^{-1})^2\cdot(1-\frac{1}{3}z^{-1})} - \frac{3}{(1 - \frac{1}{2}z^{-1})^2} - \frac{4}{(1 - \frac{1}{3}z^{-1})})\\
\]
\[
A_{11}\ =\ -6
\]
There is more generalised way to find $A_{11}$.\\\\
Let us say \textit{H(z)} is of the following form
\[
H(z)\ =\ \frac{A_{11}}{1 - \frac{1}{2}z^{-1}} + \frac{A_{21}}{(1 - \frac{1}{2}z^{-1})^2} + \frac{A_{2}}{1 - \frac{1}{3}z^{-1}}
\]
Multiply both sides by $(1 - \frac{1}{2}z^{-1})^2$, giving
\[
(1 - \frac{1}{2}z^{-1})^2\cdot H(z)\ =\ (1 - \frac{1}{2}z^{-1})\cdot A_{11} + A_{22} + A_{2}\frac{(1 - \frac{1}{2}z^{-1})^2}{(1 - \frac{1}{3}z^{-1})}
\]
Differentiate both sides w.r.t. ${z}^{-1}$ and put $(1 - \frac{1}{2}z^{-1})$ = 0 or $z$ = 2, we get
\[
-\frac{1}{2}A_{11}\ =\ \frac{d}{dz^{-1}}(1 -\frac{1}{2}z^{-1})\cdot H(z) \bigg|_{z = 2}
\]
Solving this we get, $A_{11}$ = -6.\\
This method is more general for the higher power of the factors, for example
\[
H(z)\ =\ \frac{1}{(1 - \frac{1}{2}z^{-1})^3\cdot(.....)\cdot(......)}
\]
\[
H(z) = \frac{A_{11}}{1 - \frac{1}{2}z^{-1}} + \frac{A_{21}}{(1 - \frac{1}{2}z^{-1})^2} + \frac{A_{31}}{(1 - \frac{1}{2}z^{-1})^3} +\ other \ terms
\]
Multiply both sides by $(1 - \frac{1}{2}z^{-1})^3$, we get
\[
(1 - \frac{1}{2}z^{-1})^3\cdot H(z)\ =\ (1 - \frac{1}{2}z^{-1})^2A_{11}\ +\ (1 - \frac{1}{2}z^{-1})A_{21}\ +\ A_{31}\ +\ other\ terms   
\]
Now successively differentiate w.r.t. $z^{-1}$ and put $z$ = 2.\\
So without any differentiation we get $A_{31}$, with the first differentiation we would $A_{21}$ and with the second differentiation we would get $A_{11}$.
\subsection{Differentiation w.r.t. $'z'$ : Time Shifting in the natural domain}
Let us take $(1 - \frac{1}{2}z^{-1})^{-2}$
\[
\frac{d}{dz}(\frac{1}{(1 - \frac{1}{2}z^{-1})^2})\ =\ \frac{(-2)(-\frac{1}{2}z^{-1})(-1)}{(1 - \frac{(1}{2}z^{-1})^3}
\]
\[
(-z)\frac{d}{dz}(\frac{1}{(1 - \frac{1}{2}z^{-1})^2})\ =\ \frac{2\cdot \frac{1}{2}\cdot z^{-1}}{(1 - \frac{(1}{2}z^{-1})^3}
\]
Here $'(-z)\frac{d}{dz}'$ factor corresponds to multiplication by $'n'$ in the natural domain.\\\\
And to remove $z^{-1}$ factor, replace $'n'$ by $'n+1'$ i.e. advance the sequence by 1.
\subsection{System Function as a ratio}
\begin{figure}[h]
\setlength{\unitlength}{0.14in} % selecting unit length
\centering % used for centering Figure
\begin{picture}(32,15) % picture environment with the size (dimensions)
 % 32 length units wide, and 15 units high.
\put(11,-0.5){\framebox(10,12){$ $}}
\put(5,5.5){\vector(1,0){4}}
\put(23,5.5){\vector(1,0){4}}
\put(14.5,5) {$System$}
\put(14.5,6.25) {$LSI$}
\put(14.25,7.5){$Discrete$}
\put(15,3.75) {$h[n]$}
\put(2.5,5.25) {$x[n]$}
\put(28,5.25) {$y[n]$}
\end{picture}
\end{figure}
If the following are assumed
\[
x[n]\ \xrightarrow{\ \mathcal{Z}\ }\ X(Z);\ ROC:\ R_X
\]
\[
h[n]\ \xrightarrow{\ \mathcal{Z}\ }\ H(Z);\ ROC:\ R_H
\]
\[
y[n]\ \xrightarrow{\ \mathcal{Z}\ }\ Y(Z);\ ROC:\ R_Y
\]
Then, \[y[n]\ =\ \sum\limits_{k=-\infty}^{\infty}x[k]\cdot h[n-k]\]
And by convolution property
\[Y(z)\ =\ X(z)\cdot H(z)\]
Now the system function is
\[ H(z)\ =\ \frac{Y(z)}{X(z)}\]
This ratio is independent of $X(z)$ for an LSI system. 
\subsection{Zeroes and Poles of the system}
For a rational \textit{z-transform},\\ 
the factors for which denominator is equal to zero are called the 'poles' of the system/sequence. And \\
the factors for which the numerator is zero are called the 'zeroes'  of the system/sequence.
\subsection{General process for inversion of rational \textit{z-transform}}
A rational \textit{z-transform} in $z$ or $z^{-1}$ are both equivalent.\\
For example $\frac{1}{(1 - \frac{1}{2}z^{-1})}$ and $\frac{z}{(z - \frac{1}{2})}$ are \textit{z-transforms} in $z^{-1}$ and $z$ respectively and are equal. So we can work with whichever expression we find more comfort with. Here we will work with $z^{-1}$.\\

Let us solve an example
\[
H(z)\ =\ \frac{z^2 - z}{(1 - \frac{1}{2}z^{-1})}
\]
We need to pull out the power of $z$ in the numerator.
\[
H(z)\ =\ z^2\frac{(1 - z^{-1})}{1 - \frac{1}{2}z^{-1}}
\]
$z^2$ implies that the sequence is shifted by two samples in the backward direction in the natural domain. This can be taken care of at the end of the inversion.\\\\
We now focus on the rational function of $z^{-1}$.\\\\
If degree of numerator $\geq$ degree of denominator, do a long division of the numerator w.r.t. denominator.\\\\
The quotient will give a finite series and the remainder will be the numerator of the proper fraction.
\[
\frac{(1 - z^{-1})}{1 - \frac{1}{2}z^{-1}}\ =\ 2\ +\ \frac{(-1)}{(1 - \frac{1}{2}z^{-1})}
\]
In general, a finite series can be combined with the initial power of $'z'$.\\
Suppose we have a finite series \[a_0\ +\ a_1z^{-1}\ +\ a_2z^{-2}\ +\ .........\]
And we have an initial factor of $z^D$. So we can write the complete series as
\[ S(z)\ =\ a_0z^D\ +\ a_1z^{D-1}\ +\ a_2z^{D-2}\ +\ .........\]
and this can be inverted easily
\[ s[n]\ =\ a_0\delta[n+D]\ + a_1\delta[n+(D-1)]\ +\ \delta[n+(D-2)]\ +\ ...........\] 
Hence, we get a train of discrete impulses and $s[n]$ is a finite series.\\\\
If we have a a pole at $ z = \alpha $ repeated M-fold.
\[ \frac{(........)}{(1 - \alpha z^{-1})^M\cdot(........)}\]
This contributes a 'POLYEX' term in $'n'$ of the following form
\[\Rightarrow\ (Polynomial\ in\ 'n'\ of\ degree\ 'M-1')\cdot\alpha^n\]
This 'polyex' term is associated with either $u[n]$ or $u[-n]$ or an appropriately shifted unit-step ($u[n\pm D]$).\\
This choice between $u[n]$ or $u[-n]$ comes from whether the sequence is left-sided or right-sided.\\\\
Now let us solve the above example
\[ H(z)\ =\ z^2\frac{(1 - z^{-1})}{1 - \frac{1}{2}z^{-1}}\ =\ z^2\cdot(2\ +\ \frac{(-1)}{(1 - \frac{1}{2}z^{-1})})\ =\ 2z^2\ -\ z^2\frac{1}{(1 - \frac{1}{2}z^{-1})} \]
So, by invoking the linearity property of $z-transform$, we can easily deal with the finite series and the rational proper fraction separately
\[ 2z^2\ \xrightarrow{\ \mathcal{Z}^{-1}\ }\ 2\cdot\delta[n+2]\]
\[ z^2\frac{1}{(1 - \frac{1}{2}z^{-1})};\ ROC:\ |z| > \frac{1}{2} \xrightarrow{\ \mathcal{Z}^{-1}}\ (\frac{1}{2})^{n+2}\cdot u[n+2]\]
\[ or\]
\[z^2\frac{1}{(1 - \frac{1}{2}z^{-1})};\ ROC:\ |z| < \frac{1}{2} \xrightarrow{\ \mathcal{Z}^{-1}}\ (\frac{1}{2})^{n+2}\cdot u[-n+1]\]
Therefore,
\begin{displaymath}
   h[n] = \left\{
     \begin{array}{lr}
       2\cdot\delta[n+2] + (\frac{1}{2})^{n+2}\cdot u[n+2]  & :  For\ ROC:\ |z|>\frac{1}{2}\\
       2\cdot\delta[n+2] - (\frac{1}{2})^{n+2}\cdot u[-n+1] & :  For\ ROC:\ |z|<\frac{1}{2}
     \end{array}
   \right.
\end{displaymath} 






