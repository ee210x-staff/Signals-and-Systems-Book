\section{Module 4: Lecture 13\\}

\subsection{Introduction}
In the previous lectures we have been talking about the stability of a Discrete Independent Variable System. We will be reviewing those ideas and analyze the concept of stability and causality of  
Discrete Independent Variable System using z-transforms.
We will then try to find a correlation between z-transform and Laplace transform.
\subsection{Stability and Causality of Discrete System}
 We will first discuss a general procedure to start analyzing the stability and causality of a rational system of discrete independent variables.
\begin{enumerate}
\item Look at one pole at a time
\item Identify its contribution to the impulse response
\item The contribution will be a \textit{polyex} term. Where \textit{poly} refers to the polynomial part and \textit{ex} the exponential part.
\end{enumerate}

We will now proceed by stating some lemmas and theorems in this domain.

\textbf{Lemma : } If any one of these \textit{polyex} terms has a growing exponential the impulse response is not \textit{absolutely summable}.  This can be expressed as follows, if
\begin{align*}
h_1[n] = P_1[n]\alpha^n \\
h_2[n] = P_2[n]\beta^n
\end{align*}
where, $\alpha \neq \beta$, then there do not exist any non-zero constant pairs $c_1,c_2$ which satisfy 
$c_1h_1[n] + c_2h_2[n] = 0 \forall n$\\
This analysis implies that a discrete variable rational system is stable if and only if all the \textit{polyex} terms involved have a exponential decay in the exponential part.\\
Now we know, for a pole with $|z_p| > 1$ to contribute a decaying exponential, it must  give rise to a \textit{left sided sequences}, i.e., the ROC must be interior of the pole. Similarly, for a pole with $|z_p| < 1 $ to contribute a decaying exponential, it must  give rise to a \textit{right sided sequences}, i.e., the ROC must be exterior of the pole.\\
We hence notice that the unit circle is always included. This determines the general idea, that if unit circle is in the ROC then the system is stable.

\textbf{Theorem : } A rational discrete system is stable if and only if the region of convergence of its system function includes the unit circle $|z| = 1$.\\

For a system to be causal, we are aware that the only terms that you allow are the right-sided ones, as you wish the contour $|z| \rightarrow \infty$, to lie in the ROC. Hence, the poles must lie within the unit circle for the unit circle being include in the ROC.

\textbf{Theorem : } A rational discrete system is causal and stable if and only if all its pole lie within the unit circle $|z| = 1$.\\

\subsection{Relationship between z-Transform and Laplace Transform}
We will now review some similarities in the Laplace and z-transforms. For the Laplace transform, the critical axes are $Re(s) = 0$ and $Re(s) \rightarrow \infty$, for determining stability and causality respectively. Similarly, for the z-transform, the critical contours are $|z| = 1$ and $|z| \rightarrow \infty$, for determining stability and causality respectively. \\

To relate the Laplace and z-transform, we will sample a continuous time signal $x(t)$  and compare the z-transform of $x[n]$ with the Laplace transform of $x(t)$.

\begin{align*}
x(t) \xrightarrow{\mathcal{L}} X(s), R_X
\end{align*}
Let us now sample ideally x(t) with a sampling time  $T_S$.
\begin{align*}
x(t)\sum_{n \rightarrow -\infty}^{\infty}{\delta(t - nT_S)}\\
\sum_{n \rightarrow -\infty}^{\infty}{x(nT_S)\delta(t - nT_S)}
\end{align*}
Laplace transform of the sampled signal is
\begin{align*}
&\int_{-\infty}^{\infty}{
\sum_{n \rightarrow -\infty}^{\infty}{x(nT_S)\delta(t - nT_S)}e^{-st}dt}\\
&=\sum_{n \rightarrow -\infty}^{\infty}{x(nT_S)\int_{-\infty}^{\infty}{\delta(t - nT_S)e^{-st}dt}}\\
&=\sum_{n \rightarrow -\infty}^{\infty}{x(nT_S)e^{-snT_S}}
\end{align*}

Now for the discrete sequence $x[n] = x(nT_S)$ and the z transform with $z = e^{sT_S}$, we can see that this term is exactly the same as $\sum_{n \rightarrow -\infty}^{\infty}{x[n]z^{-n}}$.\\

Hence the mapping,  $x[n] = x(nT_S)$ and $z = e^{sT_S}$, shows the relation between Laplace and z-transforms.
\begin{align*}
|z| &= |e^{sT_s}| \\
&= |e^{(\sigma + j\omega)T_S}|\\
& = |e^{\sigma T_S}|\\
& = |e^{Re(s) T_S}|\\
\end{align*}
This shows that the $Re(s)$ relates to the $|z|$ in the z-plane. Hence, vertical axes in s-plane corresponds to circle in z-plane. 
\subsection{Formal Inverse of z-Transform}
We wish to formalize our method for finding the inverse z-transform, so that we can deal with inverting non-rational functions.
\begin{align*}
x[n] \xrightarrow{z} X(z), R_X\\
X(z) = \sum_{n \rightarrow -\infty}^{\infty}{x[n]z^{-n}}
\end{align*}
where z in polar form is $z = re^{j\omega}$, hence,
\begin{align*}
X(z) = \sum_{n \rightarrow -\infty}^{\infty}{x[n]r^{-n}e^{-j\omega n}}
\end{align*}
which is the discrete time fourier transform of $x[n]r^{-n}$. Now we can invert a DTFT. 
\begin{align*}
y[n] \xrightarrow{DTFT} Y(e^{j\omega})\\
Y(e^{j\omega}) = \sum_{n \rightarrow -\infty}^{\infty}{y[n]e^{-j\omega n}}
\end{align*}
where we know $e^{j\omega n}$ are orthogonal vectors. Hence,
\begin{align*}
y[n] =\frac{1}{2\pi} \int_{-\pi}^{\pi}{Y(e^{j\omega})e^{j\omega n}d\omega }
\end{align*}

Now, we will apply this understanding for inverting the z-transform.
\begin{align*}
X(z) = \sum_{n \rightarrow -\infty}^{\infty}{x[n]r^{-n}e^{-j\omega n}}\\
x[n]r^{-n} = \frac{1}{2\pi} \int_{-\pi}^{\pi}{X(z)e^{j\omega n}d\omega }
\end{align*}
we will now replace variable of integration by substituting $z = re^{j\omega}$.
\begin{align*}
x[n] = \frac{1}{2\pi j} \oint{X(z)z^{n-1}dz }
\end{align*}
where the integral is over a circle of radius $r$ and from $-\pi$ to $\pi$, where circle of $r$ lies in the region of convergence


\subsection{Conclusion}
In this session we looked at some properties of causality and stability of discrete signals using the z-transforms. We then tried to relate the Laplace and z-transforms using sampling. We further continued our analogy by coming up with a formal expression for the inverse of the z-transform.

