\section{Module 4: Lecture 7\\Dual of the Differentiation Property and Rational Laplace Transforms}

\subsection{Introduction}
In the last lecture, we defined a specific class of Laplace transforms; the rational Laplace transforms. In this lecture, we will look in to rational Laplace transforms in greater depth. We will look at the dual of the differentiation property, by giving a formal expression for the inverse Laplace transform. Finally, we will emphasize the importance of rational Laplace transforms in the context of Linear Shift Invariant systems.
\subsection{Rational Laplace Transforms}
A rational Laplace transform is the one which can be expressed as a ratio of two \emph{finite length} series in the Laplace variable $s$, which means a ratio of a numerator finite length series in $s$, to a denominator finite length series in $s$. For example, consider the following Laplace transform:\\
\[
X(s) = \frac{1}{(s+2)(s+3)} = \frac{1}{s^2+5s+6}
\]
Note that we can have both positive and negative powers of $s$ in the series.\\
Now, as discussed in the previous lecture, we can simplify these transforms further by the use of partial fractions. So, in this particular case, we have
\[
\frac{1}{(s+2)(s+3)} = \frac{A}{(s+2)} + \frac{B}{(s+3)}
\]
where $A$ (respectively $B$) can be evaluated by multiplying this equation by $(s+2)$ (respectively $(s+3)$) and putting $s=2$ (respectively $s=3$). It is easily verified that $A=1$ and $B=-1$. Hence,
\[
\frac{1}{(s+2)(s+3)} = \frac{1}{(s+2)} - \frac{1}{(s+3)}
\]
Now, this is only the expression of the Laplace transform. To completely specify the Laplace transform, we must also specify the ROC. Note that there are three possibilities for the ROC, which is specified in the table below.
\begin{center}
  \begin{tabular}{| l | p{5cm} | p{6cm} | }
    \hline
     & Region of Convergence & Inverse Laplace Transform of $\frac{1}{(s+2)} - \frac{1}{(s+3)}$ \\ \hline \hline
    First Region & $Re(s)<-2 \ \& \ Re(s)<-3$ $\implies Re(s)<-3$ &  Take both signals \emph{left sided}. Hence $ x(t)=(-e^{-2t} +e^{-3t})u(-t)$ \\ \hline \hline
        Second Region & $Re(s)<-2 \ \& \ Re(s)>-3$ $\implies -3<Re(s)<-2$ &  $x(t) = -e^{-2t}u(-t) - e^{-3t}u(t)$ \\ \hline \hline
                Third Region & $Re(s)>-2 \ \& \ Re(s)>-3$ $\implies Re(s)>-2$&  $x(t) = e^{-2t}u(t) - e^{-3t}u(t)$ \\ 
    \hline
  \end{tabular}
\end{center}
As we know, when both $Re(s)<-2 \ $and$ \ Re(s)<-3$, both the corresponding signals will be left sided. Hence
we will get $u(-t)$ in both the cases. For the second region, as $Re(s)>-3$, the signal corresponding to $1/(s+3)$ will be right sided while the one corresponding to $1/(s+2)$ will be left sided, and similarly for the third region.\\
Now, as discussed in the earlier lecture, we can deal with terms involving higher orders of $(s+\alpha)$ in the denominator, by expanding in partial fractions accordingly and evaluate the coefficients of the various powers of $(s+\alpha)$ by successive differentiation.\\
Now, let's see why is it important to study rational Laplace transforms. For that, we have to study one more property of the Laplace transform, which, in some sense, is the dual of the differentiation property which was discussed earlier. 

\subsection{Inverse of the Laplace Transform}

We had seen what is the signal in time domain corresponding to the derivative of the Laplace transform. The question asked was as follows. If
\[
x(t) \xrightarrow{\ \mathcal{L}\ } X(s)
\]
Then what is $y(t)$ such that
\[
y(t) \xrightarrow{\ \mathcal{L}\ } \frac{\mathrm{d}X(s)}{\mathrm{d}s}
\]
It was proved that $y(t)=-t.x(t)$. Hence, differentiating in the Laplace domain results in the multiplication by the independent variable (with a minus sign) in the time domain. Similarly, we can ask the question, what will be the Laplace transform of $\frac{\mathrm{d}x(t)}{\mathrm{d}t}$?\\
To answer this question, it would be convenient if we could express $x(t)$ in terms of $X(s)$, that is, provide a formal expression for the inverse of the Laplace transform.\\
\subsubsection{Formal Inversion of the Laplace Transform}
Let us have a look at how we arrived at the Laplace transform in the first place. We first multiplied the signal by a suitable decaying exponential, so that the signal becomes absolutely integrable, and then took the Fourier transform. So
\[
x(t) \xrightarrow{ \text{multiplication by } e^{\sigma t}} x(t).e^{\sigma t} \xrightarrow{ \mathcal{F} } X(s) 
\]
where $\mathcal{F}$ denotes the Fourier transform.\\
Hence,
\[
x(t) \xrightarrow{\mathcal{L}} \int_{-\infty}^{+\infty} \! x(t)e^{\sigma t} e^{-j\Omega t} \ \mathrm{d}t
\]
Hence, we can write
\[
X(s) = \int_{-\infty}^{+\infty} \! x(t)e^{-s t} \ \mathrm{d}t \ \text{with }s=-\sigma + j\Omega
\]
Now, this itself tells us how to proceed to evaluate $x(t)$ from $X(s)$. Essentially, we have to go backwards. Consider $X(s)$ with $s = -\sigma + j\Omega$ with fixed value of $\sigma$, lying in the region of convergence. If we take its inverse Fourier transform, we should get back $x(t).e^{\sigma t}$. Hence,
\[
X(-\sigma + j\Omega) \xrightarrow{\mathcal{F}^{-1}} x(t).e^{\sigma t}
\]
Now, multiplying this output by $e^{-\sigma t}$ should give $x(t)$. Now, writing the expression for the inverse Fourier transform, we get,
\[
x(t) = \left\lbrace \frac{1}{2\pi}\int_{-\infty}^{+\infty} \! X(-\sigma + j\Omega) e^{j\Omega t} \ \mathrm{d}\Omega \right\rbrace . e^{-\sigma t}
\]
Now, since the integral is over $\Omega$, we can take the $e^{-\sigma t}$ inside the integral. So,
\[
x(t) = \frac{1}{2\pi}\int_{-\infty}^{+\infty} \! X(-\sigma + j\Omega) e^{(-\sigma + j\Omega) t} \ \mathrm{d}\Omega
\]
for a fixed $-\sigma$ lying in the region of convergence. Now, $s=-\sigma+j\Omega$. Also, as $\sigma$ is fixed, we have $\mathrm{d}s = j. \mathrm{d}\Omega$. As $\Omega$ goes from $-\infty$ to $+\infty$, $s$ goes from $-\sigma-j\infty$ to $-\sigma+j\infty$. And hence, we have our formal expression for the inverse Laplace transform:
\[
x(t) = \frac{1}{2\pi j}\int_{-\sigma-j\infty}^{-\sigma+j\infty} \! X(s)e^{s t} \ \mathrm{d}s
\]
This integral is over a contour in the complex plane in which $s$ lies. This contour is essentially a vertical line corresponding to $Re(s)=-\sigma$. Now, to simplify the notation, we will write the limits of integration as just $-\infty$ to $+\infty$, keeping in mind that the integration over the complex variable $s$ is on the vertical contour $Re(s)=-\sigma$.\\ Hence,
\[
x(t) = \frac{1}{2\pi j}\int_{-\infty}^{+\infty} \! X(s)e^{s t} \ \mathrm{d}s
\]
Now that we have obtained a formula for the inversion of the Laplace transform, we can see what happens when we differentiate $x(t)$.\\
Interestingly, the formal inversion is useful \emph{not} for inverting the Laplace transforms, in general. As we will see, in most of the cases, the inverse Laplace transform is obtained strategically by using experience. But, this formal inversion of Laplace transform can provide us a useful insight, as shown in the next subsection. 
\subsubsection{Dual of the Differentiation Property}
We had seen that if
\[
x(t) \xrightarrow{\mathcal{L}} X(s)
\]
then
\[
-t.x(t) \xrightarrow{\mathcal{L}} \frac{\mathrm{d}X(s)}{\dm s}
\]
We would like to see what happens then the signal is differentiated in the time domain instead. In short, what is the Laplace transform of $\frac{\dm x(t)}{\dm t}$?\\
We can proceed with the inversion formula obtained at the end of the previous section.
\[
x(t) = \frac{1}{2\pi j}\int_{-\infty}^{+\infty} \! X(s)e^{s t} \ \mathrm{d}s
\]
Where the integration is on a vertical line in the ROC. Let us denote the ROC by $L$.
We can now differentiate with respect to $t$ on both sides to obtain
\[
\frac{\mathrm{d}x(t)}{\mathrm{d}t} = \frac{1}{2\pi j}\frac{\mathrm{d}}{\mathrm{d}t}\int_{-\infty}^{+\infty} \! X(s)e^{s t} \ \mathrm{d}s = \frac{1}{2\pi j}\int_{-\infty}^{+\infty} \! X(s)\frac{\mathrm{d}(e^{s t})}{\mathrm{d}t} \ \mathrm{d}s = \frac{1}{2\pi j}\int_{-\infty}^{+\infty} \! s X(s)e^{s t} \ \mathrm{d}s
\]
Hence, we can see that
\[
\frac{\dm x(t)}{\dm t} = \frac{1}{2\pi j}\int_{-\infty}^{+\infty} \! \{ s X(s) \} e^{s t} \ \dm s
\]
Now, hoping that the integral would converge on the same ROC $L$, we can see that it is essentially the inverse Laplace transform of $s.X(s)$. So,
\[
\frac{\dm x(t)}{\dm t} \xrightarrow{\mathcal{L}} sX(s)
\]
Multiplication by $s$ is normally not going to change the region of convergence too much, except for possibly the extremities.\\
Hence, while differentiating in the Laplace domain results in the multiplication of the independent variable (with a minus sign) in the time domain, the differentiation in the time domain results in the multiplication of the independent variable in the Laplace domain. Hence, this is a dual property.
%\newpage
\subsubsection{Importance of the Rational Laplace Transform}
The dual of the differentiation property seen above gives us a new insight in to why the rational Laplace transforms are important. Every time you take a derivative in time you're multiplying by $s$. The other way to look at it is that every time you're multiplying by $s$, you're taking the derivative in time. That is, $sX(s)$ will be the Laplace transform of $\frac{\dm x(t)}{\dm t}$, and $s^2X(s)$ will be the Laplace transform of $\frac{\dm^2 x(t)}{\dm t^2}$, and so on. Let us see the importance of this, by looking at the following example.\\
Consider a LTI system, with an impulse response $h(t)$, having a Laplace transform. Thus, the system function is $H(s)$. So, if the input is $x(t)$ and $y(t)$, assuming both have a Laplace transform, we have,
\[
Y(s)=X(s)H(s)
\]
or
\[
H(s)=\frac{Y(s)}{X(s)}
\]
Now, consider the case when the system function is \emph{rational}. Let's say
\[
H(s)=\frac{(s+3)}{(s+2)(s+4)}
\]
Hence,
\[
\frac{(s+3)}{(s+2)(s+4)} = \frac{Y(s)}{X(s)}
\]
\[
\implies (s+3)X(s) = (s+2)(s+4)Y(s)=(s^2+6s+8)Y(s)
\]
Now, all the expressions are of the form of Laplace transform or some integral power of $s$ times the Laplace transform. Hence, using the property derived in the last sub-section, we can take the inverse Laplace transform of the equation as follows:
\[
\frac{\dm x(t)}{\dm t} + 3x(t) = \frac{\dm^2 y(t)}{\dm t^2}+6\frac{\dm y(t)}{\dm t} + 8y(t)
\]
This is a linear constant coefficient differential equation. At the end of the first module in signals and systems, we discussed that linear shift-invariant systems are most typical in natural engineering systems. Hence, we see that a for a system whose impulse response has a rational Laplace transform corresponds to a Linear Constant Coefficient Differential Equation (LCCDE) in the input and output in time domain, which occurs in most engineering systems. Here, we define a \emph{rational system} as the one which is linear and shift invariant, and has an impulse response which has a rational Laplace transform. Hence as rational systems occur most commonly, it is important to study rational Laplace transforms.\\
There is one more detail which needs to be clarified. We have
\[
x(t) \xrightarrow{\mathcal{L}} X(s)
\]
\[
\frac{\dm x(t)}{\dm t} \xrightarrow{\mathcal{L}} sX(s)
\]
What happens when there is a negative power of $s$. That is to say, which signal corresponds to the Laplace transform $\frac{X(s)}{s}$?\\
To obtain this, we must go backwards in the derivation of the dual property. Hence if we denote $\frac{\dm x(t)}{\dm t}=g(t)$ and $sX(s)=G(s)$, then we have
\[
g(t) \xrightarrow{\mathcal{L}} G(s)
\]
\[
\int_{-\infty}^t \! g(t') \ \dm t' \xrightarrow{\mathcal{L}} X(s) = \frac{G(s)}{s}
\]
Hence, in dividing by $s$, we essentially have the inverse of the derivative operator, which is called the running integral.

\subsection{Irrational Systems}
We said in the definition of rational systems, that it should be of the form numerator \emph{finite} series in $s$ upon a denominator \emph{finite} series in $s$. We have emphasised the condition of the finiteness of the series. Let us take an example of a simple irrational system whose system function can be expanded in a simple infinite series.\\
Consider a simple delay, whose system description is given by
\[
y(t)=x(t-\tau)
\]
where $\tau$ is a constant. Now, we can easily calculate the impulse response, by putting $x(t)=\delta(t)$. Hence,
\[
h(t)=\delta(t-\tau)
\]
The system function is thus
\[
H(s)=\int_{-\infty}^{+\infty} \! \delta(t-\tau)e^{-st} \ \dm t = e^{-s\tau}
\]
This is an irrational system function, as it cannot be expressed as a division of two \emph{finite} series in $s$.\\
The problem with irrational systems is that only rational systems are truly realisable, meaning that only rational systems can be realised or implemented with \emph{finite} resources. Of course, an irrational system can be in principle realised using infinite resources, but that's impractical. In practice, wherever irrational systems are needed, they are realised with a certain degree of approximation. For example, the ideal delay system as described above can be approximated by a system which behaves like a constant delay for a large range of inputs, and so on. The approximation can be improved by putting more and more resources, but to achieve the exact system, infinite resources are required.\\
Even if we cannot write a rational form of $e^{-s\tau}$, note that we can easily expand it in an infinite series, using Taylor's expansion. We have
\[
e^{-s\tau} = \sum_{n=0}^{n=\infty} \! \frac{(-s\tau)^n}{n!}
\]
where $n!$ is defined by the following equations:
\[ 0! = 1 \] \[ 1!=1 \] \[ n!=n \times (n-1)! \]
Hence, we have
\[ 2! = 2 \times 1 \] \[ 3! = 3 \times 2 \times 1 \]
and so on.
Now, this series for $e^{-s\tau}$ is infinite and \emph{cannot be converted} to a ratio of finite series. This point is stressed because there can exist infinite series which \emph{can} be expressed as a ratio of finite series. Consider, for example, the series expansion of $1/(1-s)$.
\[
\frac{1}{(1-s)} = 1+s+s^2...\infty
\]
in the region given by $|s|<1$. But the series expansion of $e^{-s\tau}$ cannot be expressed in such a form. Hence, the system is irrational.

\subsection{Conclusion}
We have seen why the rational Laplace transforms are important, through the study of the dual of the differentiation property of the Laplace transform. In the coming lectures, we will see more properties of the rational Laplace transforms, which will help us invert them, and also provide new insights into the system.
