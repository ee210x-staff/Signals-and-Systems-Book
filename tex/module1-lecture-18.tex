\section{Lecture 18: Conclusion of Module 1 \& Summary}


\subsection{Introduction}
We have studied quite a bit about Linear Shift Invariant (LSI) Systems in the previous lectures. In this lecture, we will look at what the typical Input-Output relation of an LSI system looks like.

\subsection{Typical System Description of LSI System}

If you recall, the system description of the RC circuit looked like \[ y(t) + RC\frac{d}{dt}y(t) = x(t)\]

We would like to know what the input-output relation of \emph{any} LSI system looks like, in general.

It turns out that many (continuous variable) LSI systems can be characterized by an equation of the form
\begin{alignat}{2}
y(t)& = && A_1\frac{dy(t)}{dt} + A_2\frac{d^2y(t)}{dt^2}+ \cdots + A_N\frac{d^Ny(t)}{dt^N} \nonumber \\
&+ B_0x(t)\ +\ && B_1\frac{dx(t)}{dt} + B_2\frac{d^2x(t)}{dt^2}+ \cdots + B_M\frac{d^Mx(t)}{dt^M}
\end{alignat}

where  the coefficients $A_1, A_2, \ldots A_N, B_0, B_1, B_2, \ldots B_M$ are all constants. Also, this equation is linear in all terms. Therefore, this form is called a \textbf{L}inear \textbf{C}onstant \textbf{C}oefficient \textbf{D}ifferential \textbf{E}quation (\textbf{LCCDE})

Similarly, many discrete LSI systems can be characterized by\footnote{Here we have implicitly assumed that the system is causal, but there's no reason to do so. You can have terms like $B_{-1}x[n+1]$.}:

\begin{alignat}{2}
y[n]& = && A_1y[n-1] + A_2y[n-2] + \cdots + A_Ny[n-N] \nonumber \\
&+ B_0x[n]\ +\ && B_1x[n-1] + B_2x[n-2] + \cdots + B_Mx[n-M]
\end{alignat}

This form is called a \textbf{L}inear \textbf{C}onstant \textbf{C}oefficient \textbf{D}iffer\emph{ence} \textbf{E}quation. This too is abbreviated as \textbf{LCCDE}.

We will study these LCCDEs in greater detail when we study Fourier Transforms in subsequent lectures.

\subsection{Summary of Module One}
In this module, we have looked at Discrete and Continuous Variable systems as abstractions of real-world systems. We have also studied some properties of these systems:
\begin{enumerate}
\item Additivity
\item Homogeneity
\item Shift Invariance
\item Memory
\item Causality
\item Stability (BIBO Stability)
\end{enumerate}
We then saw how Linearity (Additivity AND Homogeneity) and Shift Invariance made systems ``nice'' to study. We then looked at LSI systems in detail -- we observed that for LSI systems, we need only study the output of the system when given a special input, namely the unit impulse. We studied the unit impulse in both Continuous and Discrete contexts. We then studied the \emph{response} to this impulse, which is called the unit impulse response, in some detail. We also saw how we could predict the output of the system to any given input, using convolution.

We then tried to understand the properties of these systems in terms of this impulse response. We saw that an LSI system is causal iff\footnote{This is not a typo! \textbf{iff} stands for "if and only if".} the impulse response is absent until $t=0$. We also saw how absolute summability (for discrete systems) or absolute integrability (for continuous systems) of the impulse response \emph{guarantees} the BIBO Stability of the LSI system.

With this, we conclude module 1 of EE210.1X. In the next month or so, we will study signals and systems in a different "\emph{transformed}" domain, called the frequency domain, using the Fourier Transform.

