\section{Module 1: Lecture 4\\ System Properties: An Illustration}


\subsection{Introduction}
Recall the definitions of additivity, homegeneity and shift-invariance. In the last lecture, we saw that a system with the description, $y(t) = x(t) + 5$ is neither additive nor homogeneous. Is it shift-invariant? It is because applying the definition of shift-invariance, it can be seen that $y(t - t_{0}) = x(t - t_{0}) + 5$. In this lecture, we will consider a real-life example of a system and apply the definitions to analyse its additivity, homogeneity and shift-invariance. 

\subsection{RC Circuit}

\begin{center}
\begin{circuitikz} \draw;
begin{circuitikz}[scale=2];
    \def\xPortLeft{0}
    \def\yTerminalBottom{0}
    \def\yL{1.5}
    \def\xR{1.75}
    \def\xC{2.25}
    \def\xPortRight{3}
    % left loop
    \draw                               (\xPortLeft,\yL)
            to[R=$R$, o-]               (\xR, \yL)
            to[short]                   (\xC,\yL)
            to[C, l_=$C$,*-*]           (\xC,\yTerminalBottom)
            to[short]                   (\xPortLeft,\yTerminalBottom)
            to[voltmeter,v^>=$x(t)$,o-o]   (\xPortLeft,\yL);
    % right branch
    \draw                               (\xC,\yL)
            to[short]                   (\xPortRight,\yL)
            to[open,v^=$y(t)$,o-o]    (\xPortRight,\yTerminalBottom)
            to[short]                   (\xC,\yTerminalBottom);
\end{circuitikz}
\end{center}


Let us derive the system description for this RC circuit. Applying Kirchoff's Voltage law, we get $x(t) - i(t)R - y(t) = 0$ where $i(t) = C\frac{dy(t)}{dt}$. So we get,
\begin{equation}
x(t) = RC\frac{dy(t)}{dt} + y(t) \nonumber
\end{equation}

There is a way to check whether both additivity and homogeneity are simulatenously satisfied, using the test of \textit{superposition}.
A system is said to obey the principle of superposition if any \textit{linear combination of the inputs gives the same linear combination of the outputs}. Consider two different inputs $x_{1}(t)$ and $x_{2}(t)$, giving outputs $y_{1}(t)$ and $y_{2}(t)$ respectively. We ask whether, $\alpha x_{1}(t) + \beta x_{2}(t)$ yields $\alpha y_{1}(t) + \beta y_{2}(t)$ {\textbf for all} possible $x_{1}, x_{2}, \alpha$ and $\beta$? If so, the system obeys superposition and hence is both additive and homogeneous. Let us apply it to the RC circuit system. 
\begin{equation}\label{eq:RC1}
x_{1}(t) = RC\frac{dy_{1}(t)}{dt} + y_{1}(t)
\end{equation}
\begin{equation}\label{eq:RC2}
x_{2}(t) = RC\frac{dy_{2}(t)}{dt} + y_{2}(t) 
\end{equation}
Multiplying \eqref{eq:RC1} by $\alpha$ and \eqref{eq:RC2} by $\beta$ and adding, we get
\begin{equation}
\begin{split}
\alpha x_{1}(t) + \beta x_{2}(t) & = RC\frac{d\alpha y_{1}(t)}{dt} + \alpha y_{1}(t) + RC\frac{d\beta y_{2}(t)}{dt} + \beta y_{2}(t) \\
& = RC\frac{d (\alpha y_{1}(t) + \beta y_{2}(t))}{dt} + (\alpha y_{1}(t) + \beta y_{2}(t))
\end{split}
\end{equation}
Since $x_{1}, x_{2}, \alpha$ and $\beta$ are arbitrary, it holds for all possible values of these. Therefore, the RC system obeys principle of superpostion. 
\\

The physical interpretation of superposition is that inputs are scaled and then added one on top of the other, or in other words \textit{superposed}. If the outputs also undergo the same scaling and addition, {\textbf for all} possible scaling factors and inputs, the system obeys the principle of superposition.The principle of superposition subsumes additivity and homogeneity. That is, if the principle of superposition holds, additivity and homogeneity will both hold.

\subsubsection*{Additivity} In the principle of superposition, substitute $\alpha$ and $\beta$ both equal to 1. We get back the property of additivity. Does $x_{1}(t) + x_{2}(t)$ yield $y_{1}(t) + y_{2}(t)$ for all possible $x_{1}$ and $x_{2}$?

\subsubsection*{Homogeneity} In the principle of superposition, substitute $\beta = 0$. We get back the property of homogeneity. Does $\alpha x_{1}(t)$ yield $\alpha y_{1}(t)$ for all possible $x_{1}$ and $\alpha$?
\\

Now that it is proved that the system is superposable, let us ask the question - \textit {What changes can we make to the properties of the RC circuit system to destroy its superposability?}

\subsection{Introduction}
Recall the definitions of additivity, homegeneity and shift-invariance. In the last lecture, we saw that the RC circuit system obeys the principle of superposition. That is, it is both additive and homogeneous. In this lecture, we will see how this can be destroyed by small changes and also analyse its shift-invariance property. 

\subsection{RC Circuit}
We first start with the question : \textit {What changes can we make to the properties of the RC circuit system to destroy its superposability?} One possibility is to introduce a little nonlinearity to the system by making the response of the resistor different.
\\

For example, instead of $V \propto i$, we could have $V \propto i^{\gamma}$. This is quite true since real resistances do have a nonlinear regime. In microelectronics, resistances are made of semiconductor devices which do not have the ideal Ohm's law behaviour. Let us consider what would happen to the system in this case. Using the Kirchoff's voltage law like last time, the system description can be derived to be 
\begin{equation}
x(t) = R({C\frac{dy(t)}{dt}})^{\gamma} + y(t) \nonumber
\end{equation}
It is left as an exercise for the reader to show that this system no longer obeys superposition. This implies that the system is not \textit{both} additive and homogeneous. One must also show that the system is \textit{neither} additive nor homogeneous. 
This question has been raised to point to the fact that all real systems behave linearly over a regime and not always. 
\\

\begin{center}
	\begin{circuitikz} \draw;
		begin{circuitikz}[scale=2]
		\def\xPortLeft{0}
		\def\yTerminalBottom{0}
		\def\yL{2.0}
		\def\xR{1.75}
		\def\xC{2.25}
		\def \Vc{3.0}
		\def\xPortRight{4.0}
		\def \Vol{1.0}
		% left loop
		\draw                               (\xPortLeft,\yL)
		to[R=$R$, o-]               (\xR, \yL)
		to[short]                   (\xC,\yL)
		to[C, l_=$C$,*-]              (\xC,\Vol)
		to[battery1, v^=$2$V, -*]   (\xC, \yTerminalBottom)
		to[short]                   (\xPortLeft,\yTerminalBottom)
		to[voltmeter,v^>=$x(t)$,o-o]   (\xPortLeft,\yL);
		% right branch
		\draw                               (\xC,\yL)
		to[short]                   (\xPortRight,\yL)
		to[open,v^=$y(t)$,o-o]      (\xPortRight,\yTerminalBottom)
		to[short]                   (\xC,\yTerminalBottom);
		
		\draw                               (\xC, \yL)
		to[short]                   (\Vc,\yL)
		to[open,v^=$V_{c}(t)$,-*]      (\Vc,\Vol)
		to[short]                   (\xC,\Vol);
		
	\end{circuitikz}
\end{center}
Another way in which superposition can be destroyed is by adding a \textit{DC offset}. This can be introduced by adding a DC voltage source in series with the capacitor. Earlier, we had the capacitor voltage $V_{c}(t) = y(t)$. Now, we have $V_{c}(t) + 2 = y(t)$. The system description can be written using Kirchoff's voltage law. $x(t) - i(t)R - V_{c}(t) - 2 = 0$, where $i(t) = C\frac{dV_{c}(t)}{dt}$. So we get,
\begin{equation} \label{eq:nl}
x(t)  = RC\frac{dV_{c}(t)}{dt} + V_{c}(t) + 2 
\end{equation}
\begin{equation} \label{eq:l}
x(t) = R{C\frac{dy(t)}{dt}} + y(t) 
\end{equation}
This is the same as before and so it obeys principle of superposition. Now, if we were to consider the output to be $V_{c}(t)$ and not $y(t)$, the system description would be \eqref{eq:nl}. This is similar to the system description $y(t) = x(t) + 5$. Using the same technique as before, the reader has to show that the system described by \eqref{eq:nl} is neither additive nor homogeneous. 

\subsubsection*{Shift-invariance}
Is the system described by \eqref{eq:l} is shift-invariant? In other words, is it true that
\begin{equation}
x(t-t_{0}) = RC\frac{dy(t-t_{0})}{dt}+ y(t-t_{0}) ? \nonumber
\end{equation}
Yes. This is because the derivative operator is shift-invariant since $\frac{dy(t - t_{0})}{dt} \equiv \frac{dy({\lambda})}{d\lambda}$, where $\lambda = t - t_{0}$. 
\\

Question: Is the system described by \eqref{eq:nl} shift-invariant?







