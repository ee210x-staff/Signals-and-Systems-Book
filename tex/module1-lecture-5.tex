\section{Module 1: Lecture 5\\ Challenging Problems}


\subsection{Introduction}
So far, we have looked at systems which are either both additive and homogeneous or neither additive nor homogeneous. But they are independent properties. In this lecture, we shall look at examples of \textit{complex-valued} systems which are additive but not homogenous and vice versa. 

\subsection{Two systems}
\textit{Complex-valued} systems are those whose inputs and outputs are complex functions of time. That is, $x(t)$ and $y(t)$ take values belonging to $\mathbb{C}$ (the set of complex numbers). 
\\

\textbf{System 1 :}
\begin{equation}
y(t) = \operatorname{Re}(x(t)) \nonumber
\end{equation} 
Is this system additive? Is it homogeneous? (Remember that the constant $\alpha$ can be complex)? Is it shift-invariant? 
\\

\textbf{System 2 :}
\[
    y(t)= 
\begin{cases}
    \frac{x(t)x(t-1)}{x(t-2)},& \text{if } x(t-2) \neq 0\\
    0,              & \text{otherwise}
\end{cases}
\]
Is this system additive? Is it homogeneous? Is it shift-invariant? 
\\

The systems are, in fact, shift-invariant as shifting the input by any constant $t_{0}$ in time, will result the output to also shift by the same amount in time. What can we do to the system to make them \textit{shift-variant}? Note that shift-invariance can be destroyed by having a time-dependancy in the system description. Here is an example of explicit time-dependence. 
\begin{equation}
y(t) = tx(t) \nonumber
\end{equation} When the input is shifted by $t_{0}$, we get the output to be $tx(t-t_{0})$ which is not equal to $y(t-t_{0}) = (t-t_{0})x(t - t_{0})$. 
\\

Systems with the three properties of additivity, homogeneity and shift-invariance are very important because (a) they are easy to analyse and (b) they are easy to realise. Any electrical system consisting of ideal resistors, capacitors and inductors obeys these three properties. Mechanical systems consisting of springs and masses obey these three properties. For example, a hydraulic system can be modeled as the RC circuit that we previously saw. A tank which stores water is analogous to a capacitor which stores charge; a pipe which allows water to flow between its two ends is analogous to a resistor which allows current to pass through it. 

\textbf{Challenge} : Build a system description of a tank analogous to the capacitor using basic fluid equations. 
\textit{Hint} - Let the pressure difference between the top and bottom of the tank $P_{0}(t)$ be the output and the height of water in the tank $h(t)$ be the input. The derivative relation arises from the speed of water $v_{1}(t)$ and the instantaneous height $h(t)$ as $v_{1}(t) = \frac{dh(t)}{dt}$.
\\

\textbf{Challenge} : Build a system description of a pipe analogous to the resistor using basic fluid equations.
\textit{Hint} - Let the pressure difference across the pipe $P_{a}(t) - P_{b}(t)$ be the output and the velocity of the fluid through the pipe $v_{2}(t)$ be the input. There is a linear relationship between the two.
\\

\textbf{Challenge} : By attaching the pipe to the bottom of the tank, build a system description of the tank and pipe system analogous to the resistor-capacitor system.\newline\textit{Hint} - Let the input be the difference in the pressures between the top of the tank to the free end of the pipe. Let the output be the pressure difference across the tank. 











